\documentclass{beamer}
\usepackage[T2A]{fontenc}
\usepackage[utf8]{inputenc}
\usepackage[russian,english]{babel}

\usepackage{graphicx}
\graphicspath{}
\DeclareGraphicsExtensions{.pdf,.png,.jpg}
% Стиль презентации
\usetheme{AnnArbor}
\begin{document}
\title{Билет №11} 
\author{Комиссаров Владимир Романович}
\institute{СПбГЭТУ «ЛЭТИ»}
\date{Санкт-Петербург, 2019} 
% Создание заглавной страницы
\frame{\titlepage} 
% Автоматическая генерация содержания
\begin{frame}{Содержание}
\begin{thebibliography}{10}
\beamertemplatebookbibitems
\bibitem{1}
{\sc 1}, {\em Обработка табличных данных. Операции со строками, столбцами, формулами. Форматирование числовых данных. Построение графиков, виды графиков. Включение таблиц и графиков в текст.}.
\bibitem{2}
{\sc 2}, {\em Теоремы булевой алгебры для двух переменных. Переместительный закон. Сочетательный закон. Распределительный закон.}.
\end{thebibliography}
\end{frame}

\begin{frame}{Обработка табдичных данных}
При наборе таблиц в LaTeX, как нигде, нужна правильная организация кода - таблицы это то, что в ТеХе делать сложнее всего. И если вы свалите в кучу все управляющие метки ЛаТеХа, то потом будете проклинать всё на свете. Здесь каждый выбирает свой стиль по своему вкусу, однако надо постараться отличать управляющие символы от собственно текста таблицы.
Так как вёрстка таблиц в ЛаТеХ - дело не для слабонервных, лучше предварительно набросать на листке бумаги или графическом редакторе примерную схему таблицы. Делать наброски карандашом всегда проще, чем с воем и рыданиями переделывать код таблицы.
\end{frame}

\begin{frame}{Операции со строками, столбцами, формулами.}
- Сама по себе таблица находится в окружении bgin{tabular}{|rlc|} ... end{tabular}. Здесь вертикальная линия обозначает отделение линией столбцов, а R L и С выравнивание соответственно по правому / левому краю и по центру.
- Если столбец слишком широкий, можно задать его ширину с помощью выражения p{0.7 linewidth}, которое сделает столбец шириной в 70\% от ширины линии, например так: begin{tabular}{|p{0.4\ inewidth}p{0.4 linewidth}} сделает таблицу с двумя колонками по 40\% от ширины линии и отчеркнёт вертикальными линиями по бокам.
- Строки отделяются друг от друга при помощи команды  hline а каждая строка таблицы завершается двойным слешем \\ или командой  linebreak.
Иногда нужно объединить несколько колонок в одну на какой-то строке и убрать вертикальные линии для этого, например, чтобы сделать пояснения. В визуальных табличных редакторах типа OpenOffice.Calc это просто, но так как ЛаТеХе - это вообще говоря язык программирования, то здесь придётся немного сложнее. Для этого нужно воспользоваться командой multicolumn.
\end{frame}

\begin{frame}{Построение графиков, виды графиков}
Очень часто в документ необходимо вставить тот или иной график. На сегодняшний день есть множество инструментов позволяющие это сделать с возможностью вставки в LaTeX документ среди них Gnuplot, Matplotlib.\\
Конечно, нельзя сказать, что PGFPlots может заменить их все (например, при работе со средой R иногда удобнее положится на её собственный механизм построения графиков и просто добавлять построенные графики в документ как изображения), однако существует определенная и значительная ниша, в которой применять его удобно: учебные материалы; различные отчёты, которые будучи студентами, наверно делали все в той или иной мере; простейшая визуализация данных и т.п.

\end{frame}

\begin{frame}{Включение таблиц и графиков в текст}
В пакете longtable определены командные скобки
begin{longtable}[position]{keys}
strings
end{longtable}
для печати таблиц на нескольких страницах. Группа строк, после которой стоит
команда
endfirsthead
будет напечатана только в самом начале таблицы. Строки этой группы содержат обычно как заголовок
самой таблицы, так и заголовки колонок. Группа строк, после которой стоит команда
endhead
будет напечатана в верхней части таблицы на всех страницах, кроме первой. Группа строк, после
которой стоит команда
endfoot
будет напечатана в нижней части таблицы на всех страницах, кроме последней. И, наконец, группа
строк, после которой стоит команда
endlastfoot
будет напечатана только в самом конце таблицы.
Подпись к таблице можно напечатать командой
caption[entry]{head}

\end{frame}

\begin{frame}{Теоремы булевой алгебры для двух переменных}
На первый взгляд цифровые устройства кажутся относительно сложными. Однако они основаны на принципе многократного повторения относительно простых базовых логических схем. Инструментом такого построения служит булева алгебра, названная по имени одного из ее разработчиков – английского математика Джорджа Буля. В отличие от переменной в обычной алгебре логическая переменная имеет только два значения, которые обычно называются логическим нулем и логической единицей. В качестве обозначений используется "0" и "1" или просто 0 и 1.
Существуют три основные операции между логическими переменными: конъюнкция (логическое И), дизъюнкция (логическое ИЛИ) и инверсия (логическое НЕ). В алгебре логики используются следующие обозначения операций:\\
• конъюнкция: F = A Ù B = A • B = АВ;\\
• дизъюнкция: F = A Ú В = А + В;\\
• инверсия: F = ͞A.
\end{frame}

\begin{frame}{Сочетательный закон}
Правило.\\
Чтобы произведение двух множителей умножить на третий множитель, можно первый множитель умножить на произведение второго и третьего множителей.\\
\\
Например:\\
(7 * 6) * 5 = 7 * (6 * 5) = 210\\
(a * b) * c = a * (b * c)
\end{frame}

\begin{frame}{Переместительный закон}
Правило.\\
От перестановки множителей произведение не изменяется.\\

Например:\\
7 * 6 * 5 = 5 * 6 * 7 = 210\\
а * Ь * с = с * Ь * а
\end{frame}

\begin{frame}{Распределительный закон}
Правило.\\
Чтобы умножить число на сумму, можно умножить это число на каждое из слагаемых и полученные произведения сложить.\\

Например:\\
7 * (6 + 5) = 7 * 6 + 7 * 5 = 77\\
a * (b + c) = ab + ac
\end{frame}
\end{document}
